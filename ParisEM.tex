\documentclass[12pt]{article}
\usepackage[utf8]{inputenc}
\usepackage{amssymb}
\usepackage{makeidx}
\usepackage[english]{babel}
\usepackage{graphicx}
\usepackage{amsfonts,amsmath,amssymb,amsthm}
\usepackage{oldgerm}
\usepackage{mathrsfs}
\usepackage[active]{srcltx}
\usepackage{verbatim}
\usepackage[toc,page]{appendix}
\usepackage{aliascnt}
\usepackage{array}
\usepackage{hyperref}
\usepackage[textwidth=4cm,textsize=footnotesize]{todonotes}
\usepackage{xargs}
\usepackage{cellspace}
\usepackage[Symbolsmallscale]{upgreek}
\usepackage{geometry}
\usepackage{array}
\geometry{top=3.5cm, bottom=3.5cm, left=3.5cm , right=3.5cm}
\usepackage{fancyhdr}
\pagestyle{fancy}

\usepackage{graphicx}
\usepackage{enumerate}
\usepackage{xcolor}
\usepackage{algorithm,algorithmic}  

\newcommand{\x}[2]{x_{#1}^{(#2)}}
\newcommand{\w}[2]{w_{#1}^{(#2)}}
\newcommand{\tw}[2]{\tilde{w}_{#1}^{(#2)}}
\newcommand{\p}[2]{\xi_{#1}^{(#2)}}
\newcommand{\tp}[2]{\tilde{\xi}_{#1}^{(#2)}}
\newcommand{\ta}[2]{\tau_{#1}^{(#2)}}
\newcommand{\rmd}{\mathrm{d}}
\newcommand{\eqsp}{\;}
\newcommand{\1}{\mathrm{1}}
\newcommand{\com}[1]{{\color{gray} // #1}}
\newcommand{\mP}{\mathbb{P}}
\newcommand{\E}{\mathbb{E}}
\newcommand{\qk}{q_{k}}
\newcommand{\acom}[1]{\textit{\color{gray} //#1}}
\newcommand{\Oz}{Z}%Letter for the Ozaki approximation
\newcommand{\Jk}{J_{\alpha}^k}%Command for Jacobian of alpha
\newcommand{\mw}{\mathsf{w}}%For bridge realisations
%\newcommand{\qk}{q^{\Delta t_k}_{\theta}}
\newtheorem{lemma}{Lemma}
\newtheorem{proposition}{Proposition}

\newcounter{hypA}
\newenvironment{hypA}{\refstepcounter{hypA}\begin{itemize}
\item[{\bf H\arabic{hypA}}]}{\end{itemize}}

\begin{document}

\author{Pierre Gloaguen\footnotemark[1] \and Marie-Pierre Etienne\footnotemark[1] \and Sylvain Le {C}orff\footnotemark[2]}
 
\footnotetext[1]{AgroParistech, UMR MIA 518, F-75231 Paris, France.}
\footnotetext[2]{Laboratoire de Math\'ematiques d'Orsay, Univ. Paris-Sud, CNRS, Universit\'e Paris-Saclay.}


\title{Efficient online Sequential Monte Carlo smoother for partially observed stochastic differential equations}

\lhead{Gloaguen et al.}
\rhead{Particle smoother for SDE}

\maketitle


\section{Introduction}
This paper introduces a new algorithm to solve the smoothing problem for partially observed continuous time stochastic processes. In this setting, the hidden state process $(X_t)_{t\ge 0}$ is assumed to be a solution to a stochastic differential equation (SDE) and the only information available is given by noisy observations $(Y_{k})_{0\le k\le n}$ of the states $(X_k)_{0\le k\le  n}$ at some discrete time points $(t_k)_{0\le k\le n}$. The bivariate stochastic process $\{(X_{k},Y_{k})\}_{0\le k\le n}$ is a state space model such that conditional on the state sequence $(X_{k})_{0\le k\le n}$ the observations $(Y_{k})_{0\le k\le n}$ are independent and for all $0\le \ell\le n$ the conditional distribution of $Y_{\ell}$ given $\{X_{k}\}_{0\le k\le n}$ depends on $X_{\ell}$ only.

%Statistical inference for partially observed stochastic differential equations is a challenging task since some elementary quantities, such as transition probabilities, are not available explicitly.
Statistical inference for partially observed stochastic differential equations requires to solve bayesian filtering and smoothing problems, i.e. the computation of the posterior distributions of a sequence of hidden states given observations. Filtering refers to the estimation of the distributions of the hidden state $X_k$ given the observations $(Y_0,\ldots,Y_k)$ up to time $k$, while fixed-interval smoothing stands for the estimation of the distribution of sequence of states $(Z_{k},\ldots,Z_{p})$ given observations $(Y_{0},\ldots,Y_{\ell})$ with $0\le k\le p<\ell \le n$. 
These posterior distributions are crucial to compute maximum likelihood estimators of the unknown parameters defining the model using the observations $(Y_0,\ldots,Y_n)$ only.
% $\theta\mapsto \ell_{\theta}^{n}$ given by
%\begin{equation*}
%\ell_{\theta}^{n}(Y_{0:n}) = \log\left(\int p_{\theta}(x_{0:n},Y_{0:n})\,\rmd x_{0:n}\right)\eqsp,
%\end{equation*}  
%where $p_{\theta}$ is the complete data likelihood:
%\begin{equation*}
%p_{\theta}(x_{0:n},Y_{0:n}) = \chi(x_0)g_{0}(x_0)\prod^{n-1}_{k=0}\qk(x_{k},x_{k+1})g_{k+1}(x_{k+1})\eqsp.
%\end{equation*}
For instance, the E-step of the EM algorithm introduced in \cite{dempster:laird:rubin:1977} %iteratively builds a sequence of parameter estimates following two steps.
%following the two steps:
%\begin{enumerate}
%	\item {\bf E-step}: compute $\theta \mapsto Q(\theta,\theta_{p}) = \mathbb{E}_{\theta_p}\left[\log p_{\theta}(X_{0:n},Y_{0:n})\middle|Y_{0:n}\right]$, where $\mathbb{E}_{\theta}\left[\cdot\middle|Y_{0:n}\right]$ is the conditional expectation given $Y_{0:n}$ when the parameter value is $\theta$.
%	\item {\bf M-step}: choose $\theta_{p+1}$ as a maximizer of $\theta \mapsto Q(\theta,\theta_{p})$.
%\end{enumerate}
%step 
boils down to the computation of a conditional expectation of an additive functional of the hidden states given all the observations up to time $n$. 
%\begin{multline*}
%Q(\theta,\theta_{p}) = \mathbb{E}_{\theta_p}\left[\log \left(\chi(X_0)g_{0}(X_0)\right)\middle|Y_{0:n}\right] \\
%+ \sum_{k=0}^{n-1}\mathbb{E}_{\theta_p}\left[\log \left(\qk(X_{k},X_{k+1})g_{k+1}(X_{k+1})\right)\middle|Y_{0:n}\right] \eqsp.
%\end{multline*}
Similarly, by Fisher's identity, recursive maximum likelihood estimates may be computed using the gradient of the loglikelihood which can be written as the conditional expectation of an additive functional of the hidden states.
 See \cite[Chapter $10$ and $11$]{cappe:moulines:ryden:2005}, \cite{kantas:doucet:signh:2015,doucet:poyiadjis:singh:2011,lecorff:fort:2013a,lecorff:fort:2013b}
for further references on the use of these smoothed expectations of additive functionals  applied to maximum likelihood parameter inference in latent data models.

The exact computation of these expectations is usually not possible for general partially observed diffusions. In this paper, we propose to use Sequential Monte Carlo methods to approximate smoothing distributions with random particles associated with importance weights.
 \cite{gordon:salmond:smith:1993,kitagawa:1996} introduced the first particle filters and smoothers by combining importance sampling steps to propagate particles with importance resampling steps to duplicate or discard particles according to their importance weights. Unfortunately, these methods cannot be applied directly to partially observed stochastic differential equations since some elementary quantities, such as transition densities of the hidden states, are not available explicitly. The transition density may be approximated using discretization procedures such as the Euler-Maruyama method, the Ozaki discretization which proposes a linear approximation of the drift coefficient between two observations \cite{ozaki:1992,shoji:1998} or a Gaussian based approximation using Taylor expansions of the posterior mean and variance of an observation given the observation at the previous time step, \cite{kessler:1997,kessler:lindner:sorensen:2012,uchida:yoshida:2012}. Other approaches based on Hermite polynomials expansion were also introduced by \cite{ait-sahalia:1999,ait-sahalia:2002,ait-sahalia:2008} and extended in several directions recently, see \cite{li:2013} and all the references on the approximation of transition densities therein. However, even the most recent discretization based approximations of the transition densities induce a systematic bias of particle based approximations of posterior distributions, see \cite{delmoral:jacod:protter:2001}. To overcome this difficulty, \cite{fearnhead:papaspiliopoulos:roberts:2008} proposed to solve the filtering problem by combining SMC methods with an unbiased estimate of transition densities based on the generalized Poisson estimator (GPE) introduced in \cite{beskos:papaspiliopoulos:roberts:fearnhead:2006}.  
 
The only solution to solve the smoothing problem for partially observed SDE using SMC methods has been proposed in \cite{olsson:strojby:2011} and extends the fixed-lag smoother of \cite{olsson:cappe:douc:moulines:2008}. Using forgetting properties of the hidden chain, the algorithm improves the performance of \cite{fearnhead:papaspiliopoulos:roberts:2008} to approximate smoothing distributions but at the cost of a bias that does not vanish as the number of particles grows to infinity.
Approximations of the smoothing distributions may also be obtained using the Forward Filtering Backward Smoothing algorithm (FFBS) and  the Forward Filtering Backward Simulation algorithm (FFBSi) developed respectively in \cite{kitagawa:1996,huerzeler:kunsch:1998,doucet:godsill:andrieu:2000} and \cite{godsill:doucet:west:2004}. Both algorithms require first a forward pass which produces a set of particles and weights approximating the sequence of filtering distributions up to time $n$. In the case of partially observed SDE, this forward pass may be performed with the SMC steps combined with the GPE of \cite{fearnhead:papaspiliopoulos:roberts:2008}. Then, the backward pass of the FFBS algorithm keeps all the particles sampled during the forward pass and computes new weights to take into account the information brought by all the observations from time $n$ to time $0$. Instead of computing new weights, the FFBSi  algorithm samples independently backward in time particle trajectories using the particles and weights produced by the forward pass. Recently, \cite{olsson:westerborn:2016} proposed a new SMC algorithm, the particle-based rapid incremental smoother (PaRIS), to approximate on-the-fly (i.e. using the observations as they are received) smoothed expectations of additive functionals. Unlike the FFBS algorithm, the complexity of this algorithm grows only linearly with the number of particles $N$ and contrary to the FFBSi algorithm, no backward pass is required. 

In this paper, instead of using fixed-lag techniques which introduce a systematic additional bias in the approximation of conditional expectations, we extend the use of  PaRIS algorithm to partially observed SDE. The proposed algorithm allows to approximate smoothed expectations of additive functionals online and with a complexity growing only linearly with the number of particles. The crucial result of the application of PaRIS algorithm to SDE is that the accept reject mechanism ensuring the linear complexity of the procedure is still correct when the transition densities are replaced by bounded unbiased estimates. We also propose to improve this algorithm to obtain faster convergence of the sequence of Monte Carlo EM estimates, and to extend the range of diffusion processes for which SMC methods can be applied.To do so, we introduce a new class of GPE based on the Ozaki discretization of the target SDE, see \cite{ozaki:1992,shoji:ozaki:1998}. These new GPE estimators use a an order one Taylor expansion of the drift of the SDE to obtain more efficient estimates than Brownian Bridge based GPE. 

\section{Random weight PaRIS algorithm}
In this paper, $(X_t)_{t\ge 0}$ is a weak solution to the following SDE in $\mathbb{R}^d$:
\begin{equation}
\label{eq:target:sde}
X_0 = x_0\quad\mbox{and}\quad \rmd X_t = \alpha(X_t)\rmd t + \rmd W_t\eqsp,
\end{equation}
where $(W_t)_{t\ge 0}$ is a standard Brownian motion. It is assumed that $\alpha$ is of the form $\alpha(x) = \nabla_x A(x)$ where $A: \mathbb{R}^d \to \mathbb{R}$ is a twice continuously differentiable function. 
A popular choice of $A$  is $A = \log \pi/2$ leading to a Langevin diffusion which may be used to approximate a given target distribution $\pi$ on $\mathbb{R}^d$. See for instance \cite{roberts:tweedie:1996} for conditions under which this diffusion converges exponentially quickly to $\pi$. SDE where the drift is given by the gradient of a potential function have been widely used in ecology for instance, see \cite{brillinger:et:al:2011,harris:blackwell:2013,preisler:et:al:2013} and the references therein.

For all $0\le k \le n$, the distribution of $Y_k$ given $X_k$ has a density with respect to a reference measure $\lambda$ on $\mathbb{R}^m$ given by $g(X_k,Y_k) = g_k(X_k)$. 
The distribution of $X_0$ has a density with respect to $\lambda$ given by $\chi$.
For all $1\le k \le n$, the distribution of $X_{k+1} $ conditional on $X_{k}$ has a density $\qk(X_{k},\cdot)$ with respect to a reference measure $\mu$ on $\mathbb{R}^d$. 
Let $0 \leq k \leq k' \leq n$. The joint smoothing distributions of the hidden states are defined, for all measurable function $h$ on $(\mathbb{R}^d)^{k'-k + 1}$, by:
\[
\phi_{k:k'|n}[h] = \mathbb{E}\left[h(X_k,\ldots,X_{k'})\middle|Y_{0:n}\right]\eqsp.
\]
For all $0\le k\le n$, $\phi_{k} = \phi_{k:k|k}$ denote the filtering distributions. The aim of this section is to detail the extension of PaRIS algorithm to approximate expectations of the form
\begin{equation}
\label{def:addfunc}
H_{n} =  \sum_{k=0}^{n-1}\mathbb{E}\left[h_k(X_{k},X_{k+1})\middle|Y_{0:n}\right]\eqsp,
\end{equation}
when the transition density of the hidden states is not available explicitly and where $\{h_k\}_{k=0}^{n-1}$ are given functions on $\mathbb{R}^d\times \mathbb{R}^d$. The algorithm is based on the following link between the filtering and smoothing distributions for additive functionals, see \cite{olsson:westerborn:2016}:
\begin{equation}
\phi_{0:n|n}[h] = \phi_n[T_n[h]]\eqsp,\;\mbox{where}\; T_n[h](X_n) = \E\left[h(X_{0:n})\vert X_n,~Y_{0:n}\right]\eqsp.\label{eq:FFbsm:equality}
\end{equation}
The approximation of \eqref{eq:FFbsm:equality} requires first to approximate the sequence of filtering distributions. Sequential Monte Carlo methods provide an efficient and simple solution to obtain these approximations using sets of particles $\{\xi^{\ell}_k\}_{\ell=1}^N$ associated with weights $\{\omega^{\ell}_k\}_{\ell=1}^N$, $0\le k \le n$.
At time $k = 0$, $N$ particles $\{\xi^{\ell}_0\}_{\ell=1}^N$ are initialized by setting $\xi^{\ell}_0=x_0$. Then, $\xi^{\ell}_0$ is associated with the importance weights $\omega_0^{\ell} = g_0 (\xi^{\ell}_0)$. 
For any bounded and measurable function $h$ defined on $\mathbb{R}^d$, the expectation $\phi_{0}[h] $ is approximated by
\[
\phi^N_{0}[h] = \frac{1}{\Omega_0^N} \sum_{\ell=1}^N \omega_0^{\ell} h \left(\xi^{\ell}_0 \right)\eqsp, \quad \Omega_0^N:= \sum_{\ell=1}^N \omega_0^{\ell}\eqsp.
\]
Then, for $1\le k \le n$, using $\{(\xi^{\ell}_{k-1},\omega^{\ell}_{k-1})\}_{\ell=1}^N$, the auxiliary particle filter of \cite{pitt:shephard:1999} samples pairs $\{(I^{\ell}_k,\xi^{\ell}_{k})\}_{\ell=1}^N$ of indices and particles using an instrumental transition density $p_k$ on $\mathbb{R}^d\times \mathbb{R}^d$ and an adjustment multiplier function $\vartheta_k$ on $\mathbb{R}^d$. Each new particle $\xi^{\ell}_{k}$ and weight $\omega^{\ell}_k$ at time $k$ are computing following these steps:
\begin{enumerate}[-]
\item choose a particle index $I^{\ell}_k$ at time $k-1$ in $\{1,\ldots,N\}$ with probabilities proportional to $\omega_{k-1}^{j} \vartheta_k (\xi^{j}_{k-1})$, for $j$ in $\{1,\ldots,N\}$ ;
\item sample  $\xi^{\ell}_{k}$ using this chosen particle according to $\xi^{\ell}_{k} \sim p_k(\xi^{I^{\ell}_k}_{k-1},\cdot)$ ; 
\item $\xi^{\ell}_k$ is associated with:
\[
\omega^{\ell}_k := \frac{\qk(\xi_{k-1}^{I^{\ell}_k},\xi^{\ell}_k)g_k(\xi^{\ell}_k)}{\vartheta_k(\xi^{I^{\ell}_k}_{k-1}) p_k (\xi_{k-1}^{I^{\ell}_k},\xi^{\ell}_k)}\eqsp.
\]
\end{enumerate} 
The expectation $\phi_{k}[h]$ is approximated by
\[
\phi^N_{k}[h] := \frac{1}{\Omega_t^N} \sum_{\ell=1}^N \omega_k^{\ell} h \left(\xi^{\ell}_k \right)\eqsp,\quad\Omega_k^N:= \sum_{\ell=1}^N \omega_k^{\ell}\eqsp.
\]
The PaRIS algorithm uses the same decomposition as the FFBS algorithm introduced in \cite{doucetgodsillandrieu:2000} and the FFBSi algorithm proposed by \cite{godsill:doucet:west:2004} to approximate smoothing distributions. The FFBS algorithm has a computational complexity of order $\mathcal{O}(N^{k'-k+1})$ to approximate joint smoothing distributions of the form $\phi_{k:k'|n}$ with $k<k'$ which makes it prohibitive. An important feature of the FFBS algorithm shown in \cite{mongillo:deneve:2008,cappe:2011,delmoral:doucet:singh:2010} is that it can be implemented using only a forward pass when approximating smoothed expectations of additive functions.  Instead of computing new weights in the backward pass, the FFBSi  samples backward new trajectories from time $n$ to time $0$ among all the $N^{n+1}$ trajectories which can be chosen in $\{\xi^{\ell}_k\}$, $1\le \ell\le N$ and $0\le k\le n$. Contrary to the FFBS algorithm, the FFBSi algorithm samples trajectories and may be used to approximate joint smoothing distributions without increasing its complexity.
 Nevertheless, the computation of the transition probabilities to select particles in the backward pass makes it a $\mathcal{O}(N^2)$ algorithm. In the case where the transition probability $q_k$ is upper bounded, \cite{douc:garivier:moulines:olsson:2011} introduced an acceptance rejection mechanism to implement the FFBSi algorithm with a $\mathcal{O}(N)$ complexity, see also \cite{dubarry:lecorff:2011}.
 
The PaRIS algorithm combines both the forward only version of the FFBS algorithm with the sampling mechanism of the FFBSi algorithm. 
The algorithm does not produce an approximation of the smoothing distributions but of the smoothed expectation of a fixed additive functional and thus  may be used to approximate \eqref{def:addfunc}. 
Its crucial property is that it does not require a backward pass, the smoothed expectations is computed on-the-fly with the particle filter and no storage of the particles or weights is needed. The focus is based sufficient statistics $\tau^i_k$, approximations of $T_k[h](\xi^i_k)$, for $1\le i\le N$ and $0\le k \le n$ starting with $\tau^i_0 = 0$ for all $1\le i\le N$. Let $\tilde{N}\ge 1$. Then, at each time step $0\le k \le {n-1}$ these statistics are updated according to the following steps.
\begin{enumerate}[(i)]
\item \label{it:PaRIS:filt} Run one step of a particle filter to produce $\{(\xi^{\ell}_k, \omega^{\ell}_k)\}$ for $1\le \ell \le N$.
\item \label{it:PaRIS:sampleindex} For all $1\le i \le N$, sample independently $J_{k}^{i,\ell}$ in $\{1,\ldots,N\}$ for $1\le \ell \le \widetilde N$ with probabilities $\Lambda_{k}^N(i,\cdot)$, given by
\[
\Lambda_{k}^N(i,\ell) = \frac{\omega^{\ell}_{k} \qk(\xi^{\ell}_{k},\xi_{k+1}^{i})}{\sum_{\ell=1}^N\omega^{\ell}_{k} \qk(\xi^{\ell}_{k},\xi_{k+1}^{i})}\eqsp,\quad 1\le \ell\le N\eqsp.
\]
\item \label{it:PaRIS:smooth} Set
\[
\tau^{i}_{k+1} := \frac{1}{\widetilde{N}} \sum^{\widetilde{N}}_{\ell=1} \left\{ \tau^{J_{k}^{i,\ell}}_{k} + h_{k} \left(\xi^{J_{k}^{i,\ell}}_{k}, \xi^{i}_{k+1}\right)  \right\}\eqsp.
\]
\end{enumerate}
Then, \eqref{def:addfunc} is approximated by
\[
H_n^N = \frac{1}{\Omega_n^N}\sum_{i=1}^N \omega^{i}_n \tau_n^i\eqsp.
\] 
As proved in \cite{olsson:westerborn:2016}, the algorithm is asymptotically consistent (as $N$ goes to infinity) for any fixed precision parameter $\tilde N$. However, there is a significant qualitative difference between the cases $\tilde{N} = 1$ and $\tilde{N} \geq 2$. As for the FFBSi algorithm,  when there exists $\sigma_+$ such that $0<\qk <\sigma_+$, the PaRIS algorithm may be implemented with $\mathcal{O}(N)$ complexity using the accept reject mechanism of \cite{douc:garivier:moulines:olsson:2011}.

In general situations, the PaRIS algorithm cannot be used for stochastic differential equations as $\qk$ is unknown. Therefore, the computation of the importance weights $\omega_{k}^{\ell}$ and of the acceptance ratio of \cite{douc:garivier:moulines:olsson:2011} is not tractable. Following \cite{fearnhead:papaspiliopoulos:roberts:2008,olsson:strojby:2011},
an extension of the PaRIS algorithm may be developed by replacing $\qk$  by an  estimator $\widehat{q}_k(\xi^{\ell}_{k},\xi_{k+1}^{i};\zeta_k)$, where $\zeta_k$ is a random variable such that:
\[
\mathbb{E}\left[\widehat{q}_k(\xi^{\ell}_{k},\xi_{k+1}^{i};\zeta)\middle| \mathcal{F}_{k+1}^N\right] = \qk(\xi^{\ell}_{k},\xi_{k+1}^{i})\eqsp,
\]
where 
\[
\mathcal{F}_{k+1}^N = \sigma\left\{Y_{0:k+1};(\xi^{\ell}_{u},\omega^{\ell}_{u},\tau^{\ell}_{u}),~1\le \ell\le N,~0\le u\le k+1\right\}\eqsp.
\]
In the case where $\widehat{q}_k$ may be upper bounded by a constant $\hat{\sigma}_+$, step~\eqref{it:PaRIS:sampleindex} is replaced by the acceptance-rejection mechanism given in Algorithm~\ref{alg:AR:unknownq}. Lemma~\ref{lem:AR:unbiased} states that replacing  $\qk$ by its unbiased approximation is a valid alternative to step~\eqref{it:PaRIS:sampleindex}.
\begin{lemma}
\label{lem:AR:unbiased}
Let $0\le k\le n-1$.  For all $1\le i \le N$ and all $1\le \ell \le \widetilde{N}$, the conditional probability distribution given $\mathcal{F}_{k+1}^N$ of the random variable $J_{k}^{i,\ell}$ produced by Algorithm~\ref{alg:AR:unknownq} is  $\Lambda_{k}^N(i,\cdot)$. 
\end{lemma}
\begin{proof}
See Appendix \ref{sec:append:proofs}.
\end{proof}
\begin{algorithm}[H]
\caption{Double random weight Accept-reject-based backward sampling}
\begin{algorithmic}
\FORALL{$i \in 1,\dots, N$}
\FORALL{$\ell \in 1,\dots, \widetilde N$} 
\STATE \textbf{\sc Sampling Step} \\
Sample independently $
\zeta_k$, $U\sim \mathcal{U}[0,1]$ and $J\in\{1,\ldots,N\}$ with probabilities proportional to $\{\omega_{k}^1,\dots,\omega_{k}^N\}$.
\IF{ $$U \leq \frac{\widehat{\qk}(\xi_{k}^J,\xi_{k+1}^i,\zeta_k)}{\hat{\sigma}_+},$$}
\STATE Set $J_k^{i,\ell} = J$.
\ELSE 
\STATE Return to \textbf{\sc Sampling Step}
\ENDIF
\ENDFOR
\ENDFOR
\end{algorithmic}
\label{alg:AR:unknownq}
\end{algorithm}

\section{The OzIRis algorithm}
\label{sec:PaRIS:SDE}
In \cite{olsson:strojby:2011}, $\qk$ is estimated using GPE based on a representation of $\qk$ as an expectation under the probability measure of a Brownian bridge. 
In order to improve the Monte Carlo estimation of $\qk$,  other diffusion bridges may be used as long as they can be simulated efficiently. A simple idea is to consider a diffusion obtained as an approximation of the target SDE \eqref{eq:target:sde}. In \cite{giesecke:schwenkler:2016}, the authors proposed to replace the Brownian bridge by a Brownian motion with constant drift between two consecutive observations.
As an alternative, is is also possible to follow \cite{ozaki:1992,shoji:ozaki:1998} to propose a higher order approximation of the drift term $\alpha$.  The authors used a linear approximation of the diffusion drift $\alpha$, together with a constant approximation of the volatility over each time interval $(t_{k},t_{k+1})$. 

For all $0\le k\le n-1$, let $(Z^{k}_t)_{t\ge 0}$ be governed by the following stochastic differential equation for $t\in(t_{k},t_{k+1})$:
\begin{equation}
\label{eq:SDE:Ozaki}
\Oz^k_{t_{k}} = X_{k}\quad\mbox{and}\quad\rmd \Oz^k_t = \widetilde{\alpha}_k(\Oz^k_t)\rmd t + \rmd W_t\eqsp,
\end{equation}
where $\widetilde{\alpha}_k$ is an instrumental drift. The probability density function of the transition kernel associated with $\Oz^k$ is written $\widetilde{q}_k$.
Writing $\mathbb{Q}_k$ (resp. $\mathbb{W}_{k}$) the probability measure induced by $(X_s)_{t_{k}\le s\le t_{k+1}}$  (resp. $(\Oz_s)_{t_{k}\le s\le t_{k+1}}$) on $(\mathsf{C},\mathcal{C})$, Girsanov theorem yields:
\begin{multline*}
\frac{\rmd \mathbb{Q}_k}{\rmd \mathbb{W}_{k}}(w) = \mathrm{exp}\left\{H_{k}(\mw_{t_{k+1}}) - H_{k}(\mw_{t_{k}})\right\}\\
\times \exp\left\{- \frac{1}{2}\int_{t_{k}}^{t_{k+1}}\left\{\| \alpha_{}(\mw_s)\|^2-\|\widetilde{\alpha}_k(\mw_s)\|^2+\Delta H_k(\mw_s)\right\}\rmd s\right\}\eqsp,
\end{multline*}
where $\nabla H_k = \alpha -\widetilde{\alpha}_k$ and $\Delta$ is the Laplace operator. Then, the probability measures conditioned on hitting $X_{k+1}$ at time $t_{k+1}$ satisfy:
\begin{multline}
\label{eq:Girsanov}
\frac{\rmd \mathbb{Q}_k^{X_{k+1}}}{\rmd \mathbb{W}_k^{X_{k+1}}}(\mw) = \frac{\widetilde{\qk}(X_{k},X_{k+1})}{\qk (X_{k},X_{k+1}) }\mathrm{exp}\left\{H_{k}(X_{k+1}) - H_{k}(X_{k})\right\} \\
\times \mathrm{exp}\left\{- \frac{1}{2}\int_{t_{k}}^{t_{k+1}}\left\{\|\alpha_{}(\mw_s)\|^2-\|\widetilde{\alpha}_{k}(\mw_s)\|^2+\Delta H_{k}(\mw_s)\right\}\rmd s\right\}\eqsp.
\end{multline}
\subsection{Linear approximation of the drift}
In order to improve the performance of the GPE introduced in \cite{beskos:papaspiliopoulos:roberts:fearnhead:2006,giesecke:schwenkler:2016}, the instrumental drift is chosen as the Ozaki  \cite{ozaki:1992,shoji:ozaki:1998} linear approximation of the target drift:
\begin{equation}
\label{eq:alpha:linear}
\widetilde{\alpha}_k: x\mapsto \Jk x + \alpha(X_{k}) - \Jk X_{k}\eqsp,
\end{equation}
where $\Jk$ is the Jacobian of $\alpha$ evaluated at $X_k$. In this case, the transition kernel associated with $\Oz^k$ is given in Section~\ref{sec:ornstein:bridge} and the function $H_k$ may be chosen as 
\[
H_{k}: x \mapsto A_{}(x) - \frac{1}{2}x'\Jk x- (\alpha(X_{k}) - \Jk X_{k-1})'x\eqsp.
\]
%Consider the following assumptions on the model:
%\begin{enumerate}[(a)]
%\item For all $0\le k \le n-1$, $x\mapsto \alpha_{}(x) - \alpha_{}(X_k) - \Jk(x-X_k)$ is bounded.
%\item For all $0\le k \le n-1$, $x\mapsto \mathrm{Tr}\left(J_{\alpha_{}}(x) - \Jk\right)$ is bounded, where $\mathrm{Tr}$ is the trace operator.
%\end{enumerate}
Assume that there exists $\ell_k$ such that:
\[
\mathrm{inf}_{u\in \mathbb{R}^d} \left\{\| \alpha_{}(u)\|^2-\|\widetilde{\alpha}_{k}(u)\|^2+\Delta H_{k}(u)\right\}/2\ge \ell_k\eqsp.
\]
Define
\[
\phi_{k}: x\mapsto \left\{\|\alpha_{}(x)\|^2-\|\widetilde{\alpha}_{k}(x)\|^2+\Delta H_{k}(x)\right\}/2-\ell_k\eqsp.
\]
Assume also that the model is such that there exists $\Lambda_k$ satisfying  $\mathrm{sup}_x\eqsp \phi_{k}(x)\le \Lambda_{k}$. These assumptions might seem somehow restrictive but are necessary to define the transition density estimator in the linear case. The alternative given by \cite{beskos:papaspiliopoulos:roberts:fearnhead:2006,giesecke:schwenkler:2016} when the instrumental drift is constant does not assume any upper bound for $\phi_{k}$ but requires to sample skeletons of a Brownian bridge given its minimum/maximum between $t_k$ and $t_{k+1}$. To the best of our knowledge, there is no closed from formula to sample skeletons of an Ornstein-Uhlenbeck bridges given its extremal values on an interval. Then,
\begin{multline*}
\qk(X_{k},X_{k+1}) = \widetilde{\qk}(X_{k},X_{k+1}) \eqsp\mathrm{exp}\left\{H_{k}(X_{k+1}) - H_{k}(X_{k})-\ell\Delta_k\right\}\\
\times\mathbb{E}_{\mathbb{W}_{k}^{X_{k+1}}}\left[\mathrm{exp}\left\{-\int_{0}^{ \Delta_k}\phi_{k}(\mw_s)\rmd s\right\}\right]\eqsp,
\end{multline*}
where $\mathbb{E}_{\mathbb{W}_{k}^{X_{k+1}}}$ is the expectation under the law of the bridge process defined by \eqref{eq:SDE:Ozaki} and \eqref{eq:alpha:linear}, starting at $X_{k}$ at time $0$ and ending at $X_{k+1}$ at time $\Delta_k$.
An unbiased estimator of $\qk(X_{k},X_{k+1})$ is then obtained by estimating $\mathbb{E}_{\mathbb{W}_{k}^{X_{k+1}}}\left[\mathrm{exp}\left\{-\int_{0}^{ \Delta_k}\phi_{k}(\mw_s)\rmd s\right\}\right]$. 
Following \cite{beskos:papaspiliopoulos:roberts:fearnhead:2006}, if $\omega \sim\mathbb{W}_{k}^{X_{k}}$, $\kappa$ is a random variable taking values in $\mathbb{N}$ with distribution $\mu$ and if $(U_j)_{1\le j\le \kappa}$ are independent uniform random variables on $[0,t]$, 
\begin{align*}
\mathbb{E}_{\mathbb{W}_{k}^{X_{k+1}}}\left[\mathrm{exp}\left\{-\int_{0}^{\Delta_k}\phi_{k}(\mw_s)\rmd s\right\}\right] &\\
&\hspace{-3cm}= \mathbb{E}_{\mathbb{W}_{k}^{X_{k+1}}} \left[\mathrm{exp}\left\{-\int_{0}^{\Delta_k}\phi_{k}(\mw_s)\rmd s\right\}\right]\eqsp,\\
&\hspace{-3cm}= \mathrm{exp}\left\{-\Lambda_{k} \Delta_k\right\} \mathbb{E}_{\mathbb{W}_{k}^{X_{k+1}}} \left[\mathrm{exp}\left\{ \int_{0}^{\Delta_k}(\Lambda_{k}-\phi_{k}(\mw_s))\rmd s\right\}\right]\eqsp,\\
&\hspace{-3cm}=\mathrm{exp}\left\{-\Lambda_{k}\Delta_k\right\} \mathbb{E}_{\mathbb{W}_{k}^{X_{k+1}}}\left[\sum_{p\ge 0}\frac{1}{p!}\left(\int_{0}^{\Delta_k}\Lambda_{k}-\phi_{k}(\mw_s)\rmd s\right)^p\right]\eqsp,\\
&\hspace{-3cm}=\mathrm{exp}\left\{-\Lambda_{k}\Delta_k \right\} \mathbb{E}_{\mathbb{W}_{k}^{X_{k+1}}} \left[\mathbb{E}\left[\sum_{p\ge 0}\frac{( \Delta_k)^p}{k!} \prod_{j=1}^p\left(\Lambda_{k}-\phi_{k}(\mw_{U_j})\right)\right]\right]\eqsp,\\
&\hspace{-3cm}=\mathrm{exp}\left\{-\Lambda_{k} \Delta_k\right\} \mathbb{E}_{\mathbb{W}_{k}^{X_{k+1}}}\left[ \frac{(\Delta_k)^{\kappa}}{\mu(\kappa)\kappa!}\prod_{j=1}^{\kappa}\left(\Lambda_{k}-\phi_{k}(\mw_{U_j})\right)\right]\eqsp.
\end{align*}
Interesting choices of $\mu$ are discussed in \cite{beskos:papaspiliopoulos:roberts:fearnhead:2006} and implemented in Section~\ref{sec:exp}. Writing
\[
\zeta_k = \left\{\kappa,\mw,U_1,\ldots,U_\kappa\right\}\eqsp,
\]
Algorithm~\ref{alg:AR:unknownq} is used with 
\begin{multline}
\widehat{\qk}(X_{k},X_{k+1};\zeta_k) = \widetilde{\qk}(X_{k},X_{k+1}) \eqsp\mathrm{exp}\left\{H_{k}(X_{k+1}) - H_{k}(X_{k})-\ell_k \Delta_k\right\}\\ 
\times\mathrm{exp}\left\{-\Lambda_{k}\Delta_k\right\}\frac{( \Delta_k)^{\kappa}}{\mu(\kappa)\kappa!}\prod_{j=1}^{\kappa}\left(\Lambda_{k}-\phi_{k}(\mw_{U_j})\right)\eqsp.\label{eq:unbiased:q}
\end{multline}
The choice of $\mu$ is restricted to distributions such that the associated estimator is upper bounded by an explicit constant $\hat{\sigma}_+^k$. Then, for all $0\le k \le n-1$, $1\le i\le N$, the importance weight $\omega_k^{\ell}$ is replaced by the unbiased estimator:
\begin{equation}
\label{eq:random:weight}
\widehat{\omega}_{k+1}^{i} = \frac{M^{-1}\sum_{m=1}^M \widehat{\qk}(\xi_{k}^{I^{i}_{k+1}},\xi^{i}_{k+1};\zeta_k^m)g_{k+1}(\xi^{i}_{k+1})}{\vartheta_{k+1}(\xi^{I^{i}_{k+1}}_{k}) p_{k} (\xi_{k}^{I^{i}_{k+1}},\xi^{i}_{k+1})}\eqsp,
\end{equation}
where $\{\zeta_k^i\}_{i=1}^M$ are independent copies of $\zeta_k$. The resulting Ozaki In PaRIS (OzIRis) algortihm, which provides an approximation of \eqref{def:addfunc}, is summarized in Algorithm~\ref{alg:Ozaki:PaRIS}.
\begin{algorithm}
\caption{OzIRis algorithm}
\begin{algorithmic}
\FORALL{$i \in 1,\dots, N$}
\STATE Set $\xi_0^i = X_0$, $\omega_0^i = 1/N$ and $\tau_0^i = 0$.
\ENDFOR
\FOR{$k \in 0,\dots, n-1$}
\FORALL{$i \in 1,\dots, N$}
\STATE Set $\tau_{k+1}^i=0$;
\STATE Sample $I_{k+1}^{i}$ in $\{1,\ldots,N\}$ with probabilities proportional to $\{\widehat{\omega}_{k}^1\vartheta_{k+1}(\xi_{k}^1),\dots,\widehat{\omega}_{k}^N\vartheta_{k+1}(\xi_{k}^N)\}$.
\STATE Sample $\xi_{k+1}^{i} \sim p_k(\xi_{k}^{I_{k+1}^{i}},\cdot)$.
\STATE For all $1\le m\le M$, sample independently $\zeta_k^m=(\kappa_m,\mw_m, (U_j^m)_{1\leq j\leq \kappa_m})$ with $\kappa_m\sim \mu$, $\mw_m\sim \mathbb{W}_k^{X_{k+1}}$ and $(U_j^m)_{1\leq j\leq \kappa_m}\sim \mathcal{U}[0,\Delta_k]^{\otimes \kappa_m}$.
\STATE Compute $\widehat{\omega}^{i}_{k+1}$ using equation \eqref{eq:random:weight}.
\FORALL{$\ell \in 1,\dots, \widetilde N$} 
\STATE \textbf{\sc Sampling Step} \\
Sample independently $\zeta_k$, $U\sim \mathcal{U}[0,1]$ and $J\in\{1,\ldots,N\}$ with probabilities proportional to  $\{\widehat{\omega}_{k}^1,\dots,\widehat{\omega}_{k}^N\}$.
\IF{ $$U \leq \frac{\widehat{\qk}(\xi_{k}^J,\xi_{k+1}^i,\zeta_k)}{\hat{\sigma}^k_+},$$}
\STATE Set $J_k^{i,\ell} = j$.
\STATE Update $\tau_{k+1}^i = \tau_{k+1}^i + \left(\tau^{J_k^{i,\ell}}_{k} + h_k(\xi^{J_k^{i,\ell}}_{k},\xi^i_{k+1})\right)/\tilde{N}$;
\ELSE 
\STATE Return to \textbf{\sc Sampling Step}
\ENDIF
\ENDFOR
\ENDFOR
\ENDFOR
\end{algorithmic}
\label{alg:Ozaki:PaRIS}
\end{algorithm}

\subsection{Constant approximation of the drift}
\label{sec:constantdrift}
In the case where the function $\phi_k$ cannot be upper bounded when $\widetilde{\alpha}_k$ is set as a linear approximation of $\alpha$, \cite{giesecke:schwenkler:2016} suggests to set $\widetilde{\alpha}_k$ as a constant value $\widetilde{\alpha}_k= \rho_k$. The GPE of \cite{beskos:papaspiliopoulos:roberts:fearnhead:2006} is a special case with $\rho_k = 0$. In this case, $H_k$ may be chosen as $H_k: x\mapsto A(x) - \rho_k x$. Then, following \cite{beskos:papaspiliopoulos:roberts:fearnhead:2006,olsson:strojby:2011}, it is enough to sample a pair $(L_k,U_k)$ of random variables and a Brownian bridge $\mw\sim \mathbb{W}_k^{X_{k+1}}$ (or a Brownian bridge with a constant drift $\rho_k$) such that, for all $t_k\le t\le t_{k+1}$,
\[
L_k\le \phi_k(\mw_s)\le U_k\eqsp.
\]
Sampling the bridge conditionally on its minimum and maximum is possible using Bessel bridges as noted by \cite{beskos:papaspiliopoulos:roberts:fearnhead:2006}. Then, the second term of \eqref{eq:unbiased:q} is replaced by:
\[
\mathrm{exp}\left\{-U_{k}\Delta_k\right\}\frac{( \Delta_k)^{\kappa}}{\mu(\kappa)\kappa!}\prod_{j=1}^{\kappa}\left(U_{k}-\phi_{k}(\mw_{U_j})\right)\eqsp.
\]
Here again, the choice of $\mu$ is restricted to distributions such that the associated estimator is upper bounded by an explicit constant $\hat{\sigma}_+^k$, see Section~\ref{sec:exp}.


\section{Convergence results}
Consider the following assumptions.
\begin{hypA}
\label{assum:boundmodel}
\begin{enumerate}[(i)]
\item For all $k \geq 0$ and  all $x\in \mathbb{R}^d$, $g_{k}(x) >0$.
\item $\underset{k\geq 0}{\sup}|g_{k}|_{\infty} < \infty$.
\end{enumerate}
\end{hypA}

\begin{hypA}
\label{assum:boundalgo}
$\underset{k\geq 1}{\sup}|\vartheta_k|_{\infty} < \infty$, $\underset{k\geq 1}{\sup}|p_k|_{\infty} < \infty$ and $\underset{k\geq 1}{\sup}|\widehat{\omega}_{k}|_{\infty} < \infty$, where
\[
\widehat{\omega}_{k}(x,x';z) = \frac{\widehat{\qk}(x,x';z)g_{k+1}(x')}{\vartheta_{k+1}(x) p_{k} (x,x')}\eqsp.
\]
\end{hypA}
%The following Lemma is proved in \cite[Lemma~11]{olsson:westerborn:2016}.
%\begin{lemma}
%\label{lem:smooth:rec}
%Forward smoothing
%\end{lemma}

\begin{lemma}
\label{lem:iid}
For all $0\le k \le n-1$, the random variables $\{\widehat{\omega}_{k+1}^i\tau_{k+1}^i\}_{i=1}^N$ are independent conditionally on $\mathcal{F}_k^{N}$ and%For all bounded measurable function $h$ on $\mathbb{R}^d$,
\[
\mathbb{E}\left[\tilde{\omega}^1_{k+1}\tau^{1}_{k+1}\middle| \mathcal{F}_k^{N}\right] = \left(\phi^N_{k}[\vartheta_{k+1}]\right)^{-1}\phi^N_{k}\left[\int q_{\theta}^{\Delta t_{k+1}}(\cdot,x)g_{k+1}(x)\left\{\tau_k(\cdot) + h_{k+1}(\cdot,x)\right\}\rmd x\right]\eqsp.
\]
\end{lemma}

\begin{proof}
See appendix \ref{sec:append:proofs}
\end{proof}

\begin{proposition}
Assume that H\ref{assum:boundmodel} and H\ref{assum:boundalgo} hold and that for all $1\le k\le n$, $\mathrm{osc}(h_k)<+\infty$. For all $0\le k\le n$, there exist $b_k,c_k>0$ such that for all $N\ge 1$ and all $\varepsilon\in\mathbb{R}_+^\star$,
\[
\mathbb{P}\left(\left|\phi_k^N[\tau_k] - \phi_k\left[T_kh_k\right]\right|\ge \varepsilon\right)\le b_k\exp\left(-c_kN\varepsilon^2\right)\eqsp.
\]
\end{proposition}

\begin{proof}
Write
\[
\phi_{k+1}^N[\tau_{k+1}] - \phi_{k+1}\left[T_{k+1}h_{k+1}\right] = \frac{a_N}{b_N}\eqsp,
\]
where $a_N = N^{-1}\sum_{i=1}^N \widehat{\omega}_{k+1}^i \left(\tau_{k+1}^i - \phi_{k+1}\left[T_{k+1}h_{k+1}\right]\right)$ and $b_N =N^{-1}\sum_{i=1}^N \widehat{\omega}_{k+1}^i$. By Lemma~\ref{lem:iid}, the random variables $\{\widehat{\omega}_{k+1}^i\tau_{k+1}^i\}_{i=1}^N$ are independent conditionally on $\mathcal{F}_k^{N}$ and by H\ref{assum:boundalgo},
\[
\left|\widehat{\omega}_{k+1}^i \left(\tau_{k+1}^i - \phi_{k+1}\left[T_{k+1}h_{k+1}\right]\right)\right| \le 2|\widehat{\omega}_{k+1}|_{\infty}|H_{k+1}|_{\infty}\eqsp.
\]
Therefore, by Hoeffing inequality,
\[
\mathbb{P}\left(\left|a_N - \mathbb{E}\left[a_N\middle|\mathcal{F}_k^{N}\right]\right|\ge \varepsilon\right) = \mathbb{E}\left[\mathbb{P}\left(\left|a_N - \mathbb{E}\left[a_N\middle|\mathcal{F}_k^{N}\right]\right|\ge \varepsilon\middle|\mathcal{F}_k^{N}\right)\right]\le 2\exp\left(-c_kN\varepsilon^2\right)\eqsp.
\] 
On the other hand,
\[
\mathbb{E}\left[a_N\middle|\mathcal{F}_k^{N}\right] = \left(\phi^N_{k}[\vartheta_{k+1}]\right)^{-1}\phi^N_{k}\left[\Upsilon_k\right] \eqsp,%\left(\phi^N_{k}[\vartheta_{k+1}]\right)^{-1}
\]
where
\[
\Upsilon_k(x_k) = \int q_{\theta}^{\Delta t_{k+1}}(\cdot,x)g_{k+1}(x)\left(\tau_k(x_k) + h_{k+1}(x_k,x) - \phi_{k+1}\left[T_{k+1}h_{k+1}\right]\right)\rmd x\eqsp.
\]
By \cite[Lemma~11]{olsson:westerborn:2016}, $\phi_{k}\left[\Upsilon_k\right] = 0$ which implies by Proposition~\ref{prop:filter} that 
\[
\mathbb{P}\left(\left|\mathbb{E}\left[a_N\middle|\mathcal{F}_k^{N}\right]\right|\ge \varepsilon\right)\le b_k\exp\left(-c_kN\varepsilon^2\right)\eqsp.
\]
Then,
\[
\mathbb{P}\left(\left|a_N\right|\ge \varepsilon\right) \le b_k\exp\left(-c_kN\varepsilon^2\right)\eqsp.
\] 
Similarly, as $b_N \le |\widehat{\omega}_k|_{\infty}$, by Hoeffding inequality,
\[
\mathbb{P}\left(\left|b_N - \mathbb{E}\left[b_N\middle|\mathcal{F}_k^{N}\right]\right|\ge \varepsilon\right) = \mathbb{E}\left[\mathbb{P}\left(\left|b_N - \mathbb{E}\left[b_N\middle|\mathcal{F}_k^{N}\right]\right|\ge \varepsilon\middle|\mathcal{F}_k^{N}\right)\right]\le 2\exp\left(-c_kN\varepsilon^2\right)\eqsp.
\] 
Note that
\[
\mathbb{E}\left[b_N\middle|\mathcal{F}_k^{N}\right] = \phi^N_{k}\left[\int q_{\theta}^{\Delta t_{k+1}}(\cdot,x)g_{k+1}(x)\rmd x\right]\eqsp.%\left(\phi^N_{k}[\vartheta_{k+1}]\right)^{-1}
\]
By Proposition~\ref{prop:filter},
\[
\mathbb{P}\left(\left|\mathbb{E}\left[b_N\middle|\mathcal{F}_k^{N}\right]-\phi_k\left[\int q_{\theta}^{\Delta t_{k+1}}(\cdot,x)g_{k+1}(x)\rmd x\right]\right|\ge \varepsilon\right)\le b_k\exp\left(-c_kN\varepsilon^2\right)\eqsp.
\]
The proof is completed using Lemma~\ref{lem:hoeffding:ratio}.
\end{proof}

\begin{lemma}\label{lem:hoeffding:ratio}
Assume that $a_N$, $b_N$, and $b$ are random variables defined on the same probability space such that there exist positive constants $\beta$, $B$, $C$, and $M$ satisfying
\begin{enumerate}[(i)]
    \item $|a_N/b_N|\leq M$, $\mathbb{P}$-a.s.\ and  $b \geq \beta$, $\mathbb{P}$-a.s.,
    \item For all $\epsilon>0$ and all $N\geq1$, $\mathbb{P}\left[|b_N-b|>\epsilon \right]\leq B \exp\left(-C N \epsilon^2\right)$,
    \item For all $\epsilon>0$ and all $N\geq1$, $\mathbb{P} \left[ |a_N|>\epsilon \right]\leq B \exp\left(-C N \left(\epsilon/M\right)^2\right)$.
\end{enumerate}
Then,
$$
    \mathbb{P}\left\{ \left| \frac{a_N}{b_N} \right| > \epsilon \right\} \leq B \exp{\left(-C N \left(\frac{\epsilon \beta}{2M} \right)^2 \right)} \eqsp.
$$
\end{lemma}
\begin{proof}
See \cite{douc:garivier:moulines:olsson:2011}.
\end{proof}

\section{Numerical experiments}
\label{sec:exp}
This section investigates the performance of the proposed algorithm. The different tuning parameters are:
\begin{enumerate}[i)]
\item the number of particles $N$ and the precision parameter $\widetilde{N}$;
\item the proposal transition density $p_k$ and the adjustment multiplier $\vartheta_k$ used to propagate and weight particles;
\item the discrete distribution $\mu$ used to define $\widetilde{q}_{\theta}^{\Delta t_k}$.
\end{enumerate}
The algorithm is compared to the fixed-lag particle smoother of \cite{olsson:strojby:2011}. To complete the numerical analysis, the impact of the bridge process used to estimate $q_{\theta}^{\Delta t_k}$ is also illustrated. Instead of the Ozaki discretization, the following bridges process are considered:
\begin{enumerate}[i)]
\item a bridge with constant drift as proposed in \cite{giesecke:schwenkler:2016}.
\item a guided bridge process, see \cite{}.
\end{enumerate}
\paragraph{Choice of $\mu$}
A simple choice is to set $\mu$ as a Poisson distribution with mean $\Lambda_{\theta}t$ leading to the first estimator GPE-1:
\begin{multline*}
q^{t}_{\mathrm{GPE-1},\theta}(\kappa,\omega,(U_j)_{1\le j\le \kappa},x,y) =  \widetilde{q}^{t}_{\theta}(x,y)\eqsp\mathrm{exp}\left\{H_{\theta}(y) - H_{\theta}(x)+\ell(\theta)t\right\}\Lambda_{\theta}^{-\kappa}\\
\times \prod_{j=1}^{\kappa}\left(\Lambda_{\theta}-\phi_{\theta}(\omega_{U_j})\right)\eqsp.
\end{multline*}
% An important feature of this estimator is that under the assumptions of this section it may be bounded uniformly in $x,y$, $\sup_{x,y\in\mathbb{R}^d}  q^{t}_{\mathrm{GPE-1},\theta}(\kappa,\omega,(U_j)_{1\le j\le \kappa},x,y) \le M_\theta$ where:
% \[
% M_{\theta} = \sup_{x,y\in\mathbb{R}^d} \left(\widetilde{q}^{t}_{\theta}(x,y)\mathrm{exp}\left\{ H_{\theta}(y) - H_{\theta}(x)\right\}\right)\mathrm{exp}\left\{\ell(\theta)t\right\}\eqsp.
% \]
\paragraph{Choice of bridge process}
The approximation of $q_{\theta}^{\Delta t_k}$ described in the paper may be compared to the approximation provided by other bridges with affine drift functions. Following \cite{}, let $(Y^{\beta}_t)_{t_{k-1}\le t\le t_k}$ be governed by the following stochastic differential equation for $t_{k-1}\le t\le t_k$:
\[
Y^{\beta}_{t_{k-1}} = X_{k-1}\quad\mbox{and}\quad\rmd Y^{\beta}_t = \beta\rmd t + \rmd W_t\eqsp.
\]
As suggestd by \cite{}, a guided process between $t_{k-1}$ and $t_k$ may alo be used:
\[
Y^{\mathsf{guid}}_{t_{k-1}} = X_{k-1}\quad\mbox{and}\quad\rmd Y^{\mathsf{guid}}_t = \frac{X_k - Y^{\mathsf{guid}}_t}{t_k-t}\rmd t + \rmd W_t\eqsp.
\]
Both processes lead to estimators of $q_{\theta}^{\Delta t_k}$ following the same steps as in Section~\ref{}.
% where $\beta\in\mathbb{R}^d$ is chosen by the user. Assume that $\alpha_{\theta}$ is of the form $\alpha_{\theta}(x) = \nabla_x A_{\theta}(x)$ where $A_{\theta}: \mathbb{R}^d \to \mathbb{R}$. Writing $\mathbb{Q}_{\theta}$ (resp. $\mathbb{W}^{\beta}_{\theta}$) the probability measure induced by $(X_s)_{0\le s\le t}$  (resp. $(Y_s)_{0\le s\le t}$) on $(\mathsf{C},\mathcal{C})$, Girsanov theorem yields:
% \[
% \frac{\rmd \mathbb{Q}_{\theta}}{\rmd \mathbb{W}^{\beta}_{\theta}}(w) = \mathrm{exp}\left\{H_{\theta}(\omega_t) - H_{\theta}(\omega_0) - \frac{1}{2}\int_{0}^{t}\left\{\|\alpha_{\theta}(\mw_s)-\beta\|^2+\Delta H_{\theta}(\mw_s)\right\}\rmd s\right\}\eqsp,
% \]
% where $H_{\theta}(x) = A_{\theta}(x) -\beta'x$. Then, the probability measures conditioned on hitting $y$ at time $t$ satisfy:
% \[
% \frac{\rmd \mathbb{Q}^{y}_{\theta}}{\rmd \mathbb{W}^{\beta,y}_{\theta}}(w) = \frac{\varphi_t(y-x-t\beta)}{q^{t}_{\theta}(x,y)} \mathrm{exp}\left\{H_{\theta}(y) - H_{\theta}(x) - \frac{1}{2}\int_{0}^{t}\left\{\|\alpha_{\theta}(\mw_s)-\beta\|^2+\Delta H_{\theta}(\mw_s)\right\}\rmd s\right\}\eqsp.
% \]
% Assume also that for all $\theta$, there exists $\ell(\theta)$ such that:
% \[
% \mathrm{inf}_{u\in \mathbb{R}^d} \left\{\frac{1}{2}\left(\|\alpha_{\theta}(u)-\beta\|^2+\Delta H_{\theta}(u)\right)\right\}\ge \ell(\theta)\eqsp.
% \]
% Then
% \[
% q^{t}_{\theta}(x,y) = \varphi_t(y-x-t\beta)\eqsp\mathrm{exp}\left\{H_{\theta}(y) - H_{\theta}(x)+\ell(\theta)t\right\}\mathbb{E}_{\mathbb{W}^{\beta,y}}\left[\mathrm{exp}\left\{-\int_{0}^{t}\phi_{\theta}(\mw_s)\rmd s\right\}\right]\eqsp,
% \]
% where 
% \[
% \phi_{\theta}(x) = \frac{1}{2}\left\{\|\alpha_{\theta}(x)-\beta\|^2+\Delta H_{\theta}(x)\right\}-\ell(\theta)\eqsp.
% \]
% An unbiased estimator of $q^{t}_{\theta}(x,y)$ is then obtained by estimating $\mathbb{E}_{\mathbb{W}^{\beta,y}}\left[\mathrm{exp}\left\{-\int_{0}^{t}\phi_{\theta}(\mw_s)\rmd s\right\}\right]$.


% Let $h$ be a function defined on $\{1,\ldots,N\}$,
% \begin{align*}
% \mathbb{E}\left[h(I_{\tau})\right] & = \sum_{m\ge 0}\mathbb{E}\left[h(I_m)\1_{\tau=m}\right]\eqsp,\\
% & = \sum_{m\ge 0}h(m)\left(\prod_{\ell=0}^{m-1}\mathbb{E}\left[\1_{(\mathcal{A}^k_{\ell})^c}\right]\right)\mathbb{E}\left[h(I_m)\1_{\mathcal{A}^k_{m}}\right]\eqsp,\\
% & = \sum_{m\ge}\left(\prod_{\ell=0}^{m-1}\mathbb{E}\left[1-\frac{q^{\Delta t_k}_{\mathrm{GPE-1},\theta}(\kappa_{\ell},\omega^{\kappa_{\ell}},(U^{\kappa_{\ell}}_j)_{1\le j\le \kappa_{\ell}},\xi_{k-1}^{I_\ell},\xi_k^{1})}{M_{\theta}}\right]\right)\\
% &\hspace{5cm}\times\mathbb{E}\left[h(I_m)\frac{q^{\Delta t_k}_{\mathrm{GPE-1},\theta}(\kappa_{m},\omega^{\kappa_{m}},(U^{\kappa_{m}}_j)_{1\le j\le \kappa_{m}},\xi_{k-1}^{I_m},\xi_k^{1})}{M_{\theta}}\right]\eqsp,\\
% & = \sum_{m\ge 0}\left(\mathbb{E}\left[1-\frac{q^{\Delta t_k}_{\theta}(\xi_{k-1}^{I_1},\xi_k^{1})}{M_{\theta}}\right]\right)^m\mathbb{E}\left[h(I_m)\frac{q^{\Delta t_k}_{\theta}(\xi_{k-1}^{I_m},\xi_k^{1})}{M_{\theta}}\right]\eqsp,\\
% & = \mathbb{E}\left[h(I_1)q^{\Delta t_k}_{\theta}(\xi_{k-1}^{I_1},\xi_k^{1})\right]/\mathbb{E}\left[q^{\Delta t_k}_{\theta}(\xi_{k-1}^{I_m},\xi_k^{1})\right]\eqsp,\\
% & = \sum_{\ell=1}^N \frac{h(\ell)\omega_{k-1}^{\ell}q^{\Delta t_k}_{\theta}(\xi_{k-1}^{\ell},\xi_k^{1})}{\sum_{m=1}^N\omega_{k-1}^{m}q^{\Delta t_k}_{\theta}(\xi_{k-1}^{m},\xi_k^{1})}\eqsp,
% \end{align*}
% which concludes the proof.


\appendix

\section{Proofs}
\label{sec:append:proofs}
\begin{proof}[Proof of Lemma~\ref{lem:AR:unbiased}]
%Let $\tau$ be the first time  draws are accepted in the accept-reject mechanism. For all $\ell\ge 1$, write
%\[
%\mathcal{A}^k_{\ell} = \left\{U_\ell<\widehat{q}^{\Delta t_k}(\xi_{k-1}^{I_\ell},\xi_k^{i},\Lambda_{\ell})/\sigma_+\right\}\eqsp.
%\]
%
%Let $h$ be a function defined on $\{1,\ldots,N\}$,
%\begin{align*}
%\mathbb{E}\left[h(J^{i,j}_k)\middle| \mathcal{F}_k^N\right] & = \sum_{m\ge 1}\mathbb{E}\left[h(I_m)\1_{\tau=m}\middle| \mathcal{F}_k^N\right]\eqsp,\\
%& = \sum_{m\ge 1}h(m)\left(\prod_{\ell=1}^{m-1}\mathbb{E}\left[\1_{(\mathcal{A}^k_{\ell})^c}\middle| \mathcal{F}_k^N\right]\right)\mathbb{E}\left[h(I_m)\1_{\mathcal{A}^k_{m}}\middle| \mathcal{F}_k^N\right]\eqsp,\\
%& = \sum_{m\ge 1}\left(\prod_{\ell=1}^{m-1}\mathbb{E}\left[1-\frac{\widehat{q}^{\Delta t_k}(\xi_{k-1}^{I_\ell},\xi_k^{i},\Lambda_{\ell})}{\sigma_{+}}\middle| \mathcal{F}_k^N\right]\right)\\
%&\hspace{5cm}\times\mathbb{E}\left[h(I_m)\frac{\widehat{q}^{\Delta t_k}(\xi_{k-1}^{I_m},\xi_k^{i},\Lambda_{\ell})}{\sigma_{+}}\middle| \mathcal{F}_k^N\right]\eqsp,\\
%& = \sum_{m\ge 1}\left(\mathbb{E}\left[1-\frac{q^{\Delta t_k}_{\theta}(\xi_{k-1}^{I_1},\xi_k^{1})}{\sigma_{+}}\middle| \mathcal{F}_k^N\right]\right)^{m-1}\mathbb{E}\left[h(I_m)\frac{q^{\Delta t_k}_{\theta}(\xi_{k-1}^{I_m},\xi_k^{1})}{\sigma_{+}}\middle| \mathcal{F}_k^N\right]\eqsp,\\
%& = \mathbb{E}\left[h(I_1)q^{\Delta t_k}_{\theta}(\xi_{k-1}^{I_1},\xi_k^{i})\middle| \mathcal{F}_k^N\right]/\mathbb{E}\left[q^{\Delta t_k}_{\theta}(\xi_{k-1}^{I_m},\xi_k^{i})\middle| \mathcal{F}_k^N\right]\eqsp,\\
%& = \sum_{\ell=1}^N \frac{h(\ell)\omega_{k-1}^{\ell}q^{\Delta t_k}_{\theta}(\xi_{k-1}^{\ell},\xi_k^{i})}{\sum_{m=1}^N\omega_{k-1}^{m}q^{\Delta t_k}_{\theta}(\xi_{k-1}^{m},\xi_k^{i})}\eqsp,\\
%&= \sum_{\ell=1}^N \Lambda_{k-1}^N(i,\ell)h(\ell) \eqsp,
%\end{align*}
%which concludes the proof.
The goal is to simulate realisations of  a discrete random variable $X$ having the following target distribution:
\begin{equation}
\mP(X=\ell)= p_X(\ell)=\frac{\omega_k^\ell q_k(\xi_k^\ell,\xi_{k+1})}{\sum_{\ell=1}^N\omega_k^\ell q_k(\xi_k^\ell,\xi_{k+1})},~~\ell\in \{1,\dots,N\}
\end{equation}
for some $\xi_{k+1}$ that has no importance in this result. We assume $\omega_k$ are normalized weights, i.e. $\sum_\ell \omega_k^l=1$. The candidate will therefore be sampled from $Y$, a r.v. with the following distribution:
\begin{equation}
\mP(Y=\ell)=p_Y(\ell)= \omega_k^\ell,~~\ell\in \{1,\dots,N\}
\end{equation}
Unfortunately, the $q_k$ function is unknown. Howevere, we let's suppose we can define an unbiased estimator $\hat{q}_k(\xi_k^\ell,\xi_{k+1},\zeta)$, where $\zeta$ is a random variable having th p.d.f. $p_\zeta$.  We have therefore:
$$\E_{p_\zeta}\left[\hat{q}_k(\xi_k^\ell,\xi_{k+1},\zeta)\right]=q(\xi_k^\ell,\xi_{k+1})$$
Moreover, we impose the following property over this estimator, we want that:
$$\exists \sigma_+\in \mathbb{R}\text{ such that }\forall x,y,~~\hat{q}_k(x,y,\zeta)\leq \sigma_+$$
To simulate a random variable $X\sim p_X$, we propose the following algorithm:
\begin{enumerate}
\item Sample independently $\ell\sim p_Y$, $z\sim p_\zeta$ and $u \sim \mathcal{U}[0,1]$
\item \textbf{If} 
$$u \leq \frac{\hat{q}_k(\xi_k^\ell,\xi_{k+1},z)}{\sigma_+},$$
set $x=\ell$. \textbf{Else}, return to 1).
\end{enumerate}
The lemma \ref{lem:AR:unbiased} states that $x$ is a realisation of a random variable $X\sim p_X$. Indeed
\begin{align*}
\mP\left(Y=\ell\vert U\leq \frac{\hat{q}_k(\xi_k^Y,\xi_{k+1},\zeta)}{\sigma_+}\right)&=\frac{\mP\left(U\leq \frac{\hat{q}_k(\xi_k^Y,\xi_{k+1},\zeta)}{\sigma_+}\vert Y=l \right) \mathbb{P}(Y=\ell)}{\mP\left(U\leq \frac{\hat{q}_k(\xi_k^Y,\xi_{k+1},\zeta)}{\sigma_+}\right)}\\
&=\frac{\mP\left(U\leq \frac{\hat{q}_k(\xi_k^\ell,\xi_{k+1},\zeta)}{\sigma_+} \right) \omega_k^\ell}{\mP\left(U\leq \frac{\hat{q}_k(\xi_k^Y,\xi_{k+1},\zeta)}{\sigma_+}\right)}
\intertext{We now note that}
\mP\left(U\leq \frac{\hat{q}_k(\xi_k^Y,\xi_{k+1},\zeta)}{\sigma_+}\right)&=\E_{p_\zeta}\left[\E\left[\E_{p_Y}\left[\E\left[\mP\left(U\leq \frac{\hat{q}_k(\xi_k^Y,\xi_{k+1},\zeta)}{\sigma_+}\right) \vert Y\right] \right] \vert \zeta\right]\right]\\
&=\E_{p_\zeta}\left[\sum_{\ell=1}^N w_k^l\frac{\hat{q}_k(\xi_k^\ell,\xi_{k+1},\zeta)}{\sigma_+} \right]~~(\text{As }\frac{\hat{q}_k(\xi_k^Y,\xi_{k+1},\zeta)}{\sigma_+}\leq 1)\\
&=\frac{1}{\sigma_+}\sum_{\ell=1}^N w_k^\ell q_k(\xi_k^\ell,\xi_{k+1})
\intertext{In the same way, we have:}
\mP\left(U\leq \frac{\hat{q}_k(\xi_k^\ell,\xi_{k+1},\zeta)}{\sigma_+}\right)&=\frac{1}{\sigma_+} q_k(\xi_k^\ell,\xi_{k+1})
\intertext{Which gives overall the wanted result}
\mP\left(Y=\ell\vert U\leq \frac{\hat{q}_k(\xi_k^Y,\xi_{k+1},\zeta)}{\sigma_+}\right)&=\frac{\omega_k^\ell q_k(\xi_k^\ell,\xi_{k+1})}{\sum_{l=1}^N w_k^\ell q_k(\xi_k^\ell,\xi_{k+1})}=p_X(\ell),~~\ell\in \{1,\dots,N\}
\end{align*}
%\end{proof}
\end{proof}
\begin{proof}[Proof of Lemma \ref{lem:iid}]
The independence is ensured by the mechanism of SMC methods.\\
By \eqref{eq:random:weight},
\[
\mathbb{E}\left[\tilde{\omega}^i_{k+1}\tau^{i}_{k+1}\middle| \mathcal{F}_k^{N}\right] = \mathbb{E}\left[\frac{M^{-1}\sum_{m=1}^M \widehat{\qk}(\xi_{k}^{I^{i}_{k+1}}, \xi^{i}_{k+1};\zeta^m_{k+1})g_{k+1}(\xi^{i}_{k+1})}{\vartheta_{k+1}(\xi^{I^{i}_{k+1}}_{k}) p_{k+1}(\xi_{k}^{I^{i}_{k+1}},\xi^{i}_{k+1})}\tau^{i}_{k+1}\middle| \mathcal{F}_k^{N}\right]\eqsp.
%&= \mathbb{E}\left[\frac{g_{k+1}(\xi^{1}_{k+1})}{\vartheta_{k+1}(\xi^{I^{1}_{k+1}}_{k}) p_{k+1}(\xi_{k}^{I^{1}_{k+1}},\xi^{\ell}_{k+1})}\tau^{1}_{k+1}\widehat{q}_{\theta}^{\Delta t_{k+1}}(\xi_{k}^{I^{1}_{k+1}},\xi^{1}_{k+1};\Lambda_{k+1})\middle| \mathcal{F}_k^{N}\right]\eqsp,
\]
Note that
\begin{align*}
&\mathbb{E}\left[\tau^{i}_{k+1}\middle|\sigma\left(\mathcal{F}_k^{N},\{\xi_{k+1}^i,I_{k+1}^i\}\right)\right]
 = \sum_{\ell=1}^N\frac{\omega_k^{\ell} \qk(\xi_{k}^{\ell}, \xi^{i}_{k+1}) \left(\tau^{\ell}_k + h_{k}(\xi_{k}^{\ell},\xi^{i}_{k+1})\right)}{\sum_{\ell'=1}^N\omega_k^{\ell'} \qk(\xi_{k}^{\ell'},\xi^{i}_{k+1})}\eqsp,\\
&\mathbb{E} \left[ M^{-1}\sum_{m=1}^M \widehat{\qk}(\xi_{k}^{I^{i}_{k+1}},\xi^{i}_{k+1};\zeta^m_{k+1}) \middle| \sigma \left(\mathcal{F}_k^{N},\{\xi_{k+1}^i,I^{i}_{k+1}\}\right)\right]
 = \qk(\xi_{k}^{I^{i}_{k+1}},\xi^{i}_{k+1})\eqsp.
\end{align*}
Since $\tau^{i}_{k+1}$ and $\{\zeta^m_{k+1}\}_{i=1}^M$ are independent conditionally to $\sigma\left(\mathcal{F}_k^{N},\{\xi_{k+1}^l, I_{k+1}^l\}_{l=1}^N\right)$:
\begin{multline*}
\mathbb{E}\left[\tau^{i}_{k+1}M^{-1}\sum_{m=1}^M \widehat{\qk} (\xi_{k}^{I^{i}_{k+1}},\xi^{i}_{k+1};\zeta^m_{k+1})\middle|\sigma\left(\mathcal{F}_k^{N},\{\xi_{k+1}^i,I_{k+1}^i\} \right)\right]\\
 = q_k(\xi_{k}^{I^{i}_{k+1}},\xi^{i}_{k+1})\sum_{\ell=1}^N\frac{\omega_k^{\ell} \qk (\xi_{k}^{\ell},\xi^{i}_{k+1})\left(\tau^{\ell}_k + h_{k}(\xi_{k}^{\ell},\xi^{i}_{k+1})\right)}{\sum_{\ell'=1}^N\omega_k^{\ell'} \qk (\xi_{k}^{\ell'},\xi^{i}_{k+1})}\eqsp.
\end{multline*}
Moreover, The p.d.f. of the couple $(\xi_{k+1}^i,I_{k+1}^i)\vert \mathcal{F}_k^N$ is given by
\begin{align*}
p(x,j) &= \frac{\omega_k^j\vartheta_{k+1}(\xi_k^j)p_k(\xi_k^j,x)}{\sum_{l=1}^N\int \omega_k^l\vartheta_{k+1}(\xi_k^l)p_k(\xi_k^l,y)\rmd y}=\frac{\omega_k^j\vartheta_{k+1}(\xi_k^j)p_k(\xi_k^j,x)}{\phi_k^N[\vartheta_{k+1}]}
\end{align*}
Therefore, this yields:
\begin{align*}
\mathbb{E}\left[\tilde{\omega}^i_{k+1}\tau^{i}_{k+1}\middle| \mathcal{F}_k^{N}\right]&= \left(\phi^N_{k}[\vartheta_{k+1}]\right)^{-1} \sum_{j=1}^N\omega_k^j \int \vartheta_{k+1}(\xi^{j}_{k})\frac{\qk(\xi_{k}^{j},x) g_{k+1}(x)}{\vartheta_{k+1}(\xi^{j}_{k}) p_{k}(\xi_{k}^{j},x)}\\
&\hspace{1cm}\times \sum_{\ell=1}^N\frac{\omega_k^{\ell} \qk (\xi_{k}^{\ell},x)\left(\tau^{\ell}_k + h_{k}(\xi_{k}^{\ell},x)\right)}{\sum_{\ell'=1}^N\omega_k^{\ell'}\qk(\xi_{k}^{\ell'},x)}p_{k}(\xi_{k}^{j},x)\rmd x\eqsp,\\
&= \left(\phi^N_{k}[\vartheta_{k+1}]\right)^{-1}\\
&~~~~\times\sum_{\ell=1}^N \omega_k^\ell\left[\int \frac{ \sum_{j=1}^N \omega_k^j\qk(\xi_k^j,x) }{ \sum_{\ell'=1}^N\omega_k^{\ell'}\qk(\xi_{k}^{\ell'},x) } g_{k+1}(x)\qk (\xi_{k}^{\ell},x)\left(\tau^{\ell}_k + h_{k}(\xi_{k}^{\ell},x)\right) \rmd x \right]\\ 
& =\left(\phi^N_{k}[\vartheta_{k+1}]\right)^{-1}\phi^N_{k}\left[\int \qk(\cdot,x)g_{k+1}(x)\left\{\tau_k(\cdot) + h_{k}(\cdot,x)\right\}\rmd x\right]\eqsp,
\end{align*}
which concludes the proof.
\end{proof}

\section{Multidimensional Ornstein Uhlenbeck bridge}
\label{sec:ornstein:bridge}
The GPE introduced in this paper requires to sample finite dimensional distributions of Ornstein Uhlenbeck bridges. These results may be found in \cite{} and are given here for completeness. Consider the following $d$-dimensional SDE:
\begin{equation}
\label{eq:sde:ornstein}
Z_0 = z_0\quad\mbox{and}\quad \rmd Z_t = (\Upsilon Z_t + \vartheta)\rmd t + \rmd W_t\eqsp,
\end{equation}
where $W$ is a standard Brownian motion on $\mathbb{R}^d$, $A$ is a $d\times d$ matrix and $b$ a $d$ dimensional vector. In the case of Section~\ref{sec:PaRIS:SDE}, for all $1\le k \le n$, the proposal stochastic differential equation governing $Y_t^k$, $t\in(t_{k-1},t_k)$, is of the form \eqref{eq:sde:ornstein} with
\[
\Upsilon = J_{\alpha_{\theta}}(X_{k-1})\;\;\mbox{and}\;\; \vartheta =\alpha_{\theta}(X_{k-1}) - J_{\alpha_{\theta}}(X_{k-1})X_{k-1}\eqsp,
\]
$J_{\alpha_{\theta}}$ being the Jacobian of $\alpha_{\theta}$. In this case, as $\alpha_{\theta}(x) = \nabla_x A_{\theta}(x)$, $A$ is symmetric and nonsingular. By \cite[Section~5.6]{karatzas:shreve:1991}, the unique strong solution of \eqref{eq:sde:ornstein} is:
\[
Z_t = \mathrm{e}^{t\Upsilon}z_0 + \int_0^t\mathrm{e}^{(t-s)\Upsilon}\vartheta \rmd s + \int_0^t\mathrm{e}^{(t-s)\Upsilon} \rmd W_s\eqsp.
\]
Therefore, the probability density of the conditional distribution of $Z_t$ given $Z_0 = z_0$ is Gaussian with mean $\mu_t(z_0)$ and variance $\Sigma_{t}$ given by:
\begin{align*}
\Sigma_{t} &= \int_0^{t} \mathrm{e}^{(t -s)\Upsilon}\mathrm{e}^{(t -s)\Upsilon'}\rmd s\eqsp,\\
\mu_{t}(z) &= \mathrm{e}^{t \Upsilon}z_0 + \left(\int_0^t\mathrm{e}^{(t-s)\Upsilon} \rmd s\right)\vartheta\eqsp.
\end{align*}
As $\Upsilon$ is symetric and nonsingular,
\begin{align*}
\Sigma_{t} &= \int_0^{t} \mathrm{e}^{2(t -s)\Upsilon}\rmd s = \Upsilon^{-1}(\mathrm{e}^{2t\Upsilon}-I_d)/2\eqsp,\\
\mu_{t}(z) &= \mathrm{e}^{t \Upsilon}z + \left(\int_0^t\mathrm{e}^{(t-s)\Upsilon} \rmd s\right)\vartheta = \mathrm{e}^{t \Upsilon}z + \Upsilon^{-1}(\mathrm{e}^{t\Upsilon}-I_d)\vartheta \eqsp.
\end{align*}
For each interval $(t_{k-1},t_k)$, $1\le k \le n$, the GPE requires to sample a Ornstein-Uhlenbeck bridge at some ramdom time steps between $X_{k-1}$ at time $t_{k-1}$ and $X_k$ at time $t_k$. Conditional on $Z_s = x$ and $Z_T = y$, the probability density of $Z_t$ is Gaussian with mean $\mu$ and variance $\Sigma_t^{s,T}$ given by: 
\begin{align*}
\mu &=\Gamma(t,T)\Gamma(s,T)^{-1}\mu_- + \Gamma(s,t)'(\Gamma(s,T)')^{-1}\mu_+\eqsp,\\
\Sigma_t^{s,T} &=\Gamma(t,T)\Gamma(s,T)^{-1}\Gamma(s,t)\eqsp,\\
\Gamma(s,t)&=\int_s^{t} \mathrm{e}^{(s -u)\Upsilon}\mathrm{e}^{(t -u)\Upsilon'}\rmd u\eqsp = \Upsilon^{-1}\left(\mathrm{e}^{(t -s)\Upsilon}-\mathrm{e}^{(s -t)\Upsilon}\right)/2\eqsp,\\
\mu_- &= x - \Upsilon^{-1}(\mathrm{e}^{(s -t)\Upsilon}-I_d)\vartheta\eqsp,\\
\mu_+ &= y-\Upsilon^{-1}(\mathrm{e}^{(T -t)\Upsilon}-I_d)\vartheta\eqsp.
\end{align*}


\bibliographystyle{plain}
\bibliography{./ParisEM_bib}
\end{document}